\chapter{Background}
Internet connectivity is based on a straightforward idea: endpoints exchange packets using globally reachable addresses. For most of the Internets history, 
those addresses have been IPv4, and the surrounding ecosystem: routing practices, DNS operations, firewall policies, and application behavior: grew up around that assumption. As networks expanded and diversified, administrators turned to private addressing and NAT to stretch limited IPv4 space. Over time these workarounds stopped being exceptional and became part of the baseline, with many applications implicitly depending on NAT and other middleboxes rather than a clean end-to-end model.
The limits of this approach are structural. IPv4 offers a 32-bit address space of about 4.3 billion addresses, and a noticeable amount is reserved for private use, special purposes, or infrastructure. To keep growth possible, operators widely adopted NAT44 and eventually carrier-grade NAT, centralizing state and rewriting addresses to multiply supply. These techniques do conserve addresses, but they introduce edge cases at the protocol level and make debugging or policy enforcement more complex.
It is worth addressing a common question: what happened to “IPv5”? The name informally refers to the experimental Internet Stream Protocol (ST and later ST2), which used protocol number 5 in the IP header. It was designed for connection-oriented streams and quality-of-service experiments, not as a general replacement for IPv4. Importantly, it retained the 32-bit addressing model and never achieved wide deployment on the public Internet. When the IETF set out to design the next general-purpose Internet Protocol, it skipped over the “5” label already associated with ST and standardized IPv6 as the successor to IPv4.
IPv6 was designed to remove IPv4's constraints rather than to extend them incrementally. Its 128-bit address space $($ $2^{128}$ addresses, or  roughly 340 undecillion$)$ exceeds IPv4s 32-bit space, an increase by a factor of $2^{96}$. The protocol revises addressing and neighbor discovery and in typical deployments avoids the need for NAT. Even so, the shift cannot happen overnight. The global free pools of IPv4 addresses were exhausted at different times—IANA in 2011, ARIN in 2015, and RIPE NCC in 2019—and IPv6 enablement has progressed unevenly across networks, content providers, and regions. As a result, the Internet has been living in a coexistence phase. Many networks and services speak IPv6, yet IPv4-only systems are still common enough that compatibility must be maintained. The real-world effects come down to two things: IPv4 addresses are running out, and the shift to IPv6 is happening slowly.

\section{IPv4 exhaustion and transition to IPv6}
Internet use has expanded to critical infrastructure across consumer, enterprise, and public sectors. Services running at any time of the day and the increase of connected devices have driven steady growth in traffic and endpoints, making address management a central concern\cite{7737362,LEVIN20141059}. IP addresses perform two fundamental roles: identifying endpoints and enabling packet delivery across networks, and uniqueness at global scale is essential\cite{LEVIN20141059}.
IPv4, standardized in 1981-1983, provides a 32-bit address space of roughly 4.3 billion addresses\cite{rfc791}.In practice, not all addresses are usable on the public Internet due to special-use and private allocations, and early design choices further reduced the effectively usable pool\cite{rfc1918,7737362,LEVIN20141059}. Management techniques such as CIDR and NAT slowed the pace of consumption but could not eliminate the finite limit\cite{7737362}.
Exhaustion means that the pool of unused IPv4 addresses has run out, not that IPv4 connections themselves have stopped working. The process started at the global level and then moved downward: IANA allocated its final IPv4 blocks on 3 February 2011, after that, each Regional Internet Registry (RIR) moved into its final phase: APNIC in April 2011, RIPE NCC in September 2012, LACNIC in June 2014 while AFRINIC held on the longest\cite{LEVIN20141059}. A global policy passed on 6 May 2012 established mechanisms for the recovery and redistribution of returned IPv4 address space, yet scarcity has continued to persist\cite{7737362}. The impact has been uneven across regions because historical allocations left some operators with far fewer addresses per user than others\cite{LEVIN20141059}.
The shortage was unavoidable because demand kept rising: by 2014, about 2.9 billion people were online, with more than 200 million new users joining each year after 2010. Internet use jumped from less than 1% of the world’s population in 1994 to over 40% by 2014, and surpassing 4 billion by 2020 and 5.5 billion in 2024 \cite{7737362, itu_d_statistics}. In short, exponential demand confronted a finite IPv4 pool.

IPv6 was standardized in 1995 as the long-term successor, expanding addressing to 128 bits—on the order of $3.4 \times 10^{38}$ unique addresses—and introducing protocol-level improvements aimed at routing scalability, mobility support, and operational security\cite{rfc1883,7737362,LEVIN20141059}. The allocation data highlights how abundant IPv6 is compared to IPv4, even when large IPv6 distributions are considered\cite{7737362}. However, IPv4 and IPv6 don't work together on their own, so bridging mechanisms are needed while both remain in use\cite{LEVIN20141059}.
Even though the technical benefits of IPv6 were clear, its adoption lagged behind the urgency created by IPv4 shortages. IPv6 allocations reported by the RIRs lagged behind user growth, with especially low adoption in some regions, like Africa. At the same time, user-side measurements, such as those from Google, showed less than 10% IPv6 usage, though the trend was gradually increasing \cite{7737362}. Operators had to cover costs for enabling IPv6, including new equipment and configuration, which extended the period of running both protocols, often well beyond 2020 and in some networks possibly past 2030 \cite{7737362}.
Operators and policymakers have followed three main approaches: making better use of IPv4, allocating the remaining IPv4 space more efficiently and transitioning to IPv6\cite{LEVIN20141059}. On the technical side, NAT, especially carrier-grade NAT, conserves public IPv4 addresses by multiplexing many private hosts behind a smaller set of public addresses\cite{rfc2663}. By around 2014, a measurable fraction of users were estimated to traverse CGN\cite{livadariu2018inferring}. While NAT is effective at saving IP addresses, it can make end-to-end connectivity more complicated and add to operational complexity\cite{rfc2993}. Dual stack (running IPv4 and IPv6 in parallel) is broadly available and will remain common for years, but it does not address the decline of IPv4\cite{LEVIN20141059}. Policy measures like smaller allocations and efforts to reclaim unused addresses have helped ease IP address scarcity somewhat, but they don't eliminate the need for IPv6\cite{LEVIN20141059}.
Strategically this means that IPv4 and IPv6 will continue to coexist for a long time, connected through integration methods\cite{7737362,LEVIN20141059}. This situation drives the focus of this thesis: as network operators decide whether to keep dual-stack setups or move to IPv6-only access with translation technologies (like NAT64/DNS64, often paired with CLAT on the client side), it's important to understand the performance trade-offs involved. The next section takes a look at the main transition mechanisms that form the basis of this comparison\cite{7737362,LEVIN20141059}.

\section{IPv6 transition mechanisms}

\section{Software implementations of NAT64}