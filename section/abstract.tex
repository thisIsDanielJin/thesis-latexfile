\section*{Abstract}
IPv4 address exhaustion has accelerated the shift toward IPv6-first networks that must still maintain reachability to IPv4-only services. While Dual-Stack remains the prevailing deployment model, translation-based designs, NAT64 with DNS64 and client-side translation via CLAT as in 464XLAT, offer an alternative that can reduce IPv4 dependency. This thesis quantifies the performance trade-offs between Dual-Stack and CLAT by measuring TCP throughput and ICMP round-trip time (RTT) under realistic Linux implementations and topologies. Three translators: Jool (kernel-space SIIT/NAT64), Tayga (stateless user space), and Tundra (multi-threaded user space SIIT/CLAT), were evaluated across three environments: an AWS setup, a bare-metal Ubuntu host, and a bare-metal network connected via Ethernet. The experiments used Linux network namespaces to control topology and isolate components and included native IPv6 baselines with one and two hops to separate hop-count effects from translation cost.
The results show that CLAT introduces only a small performance overhead relative to Dual-Stack for the tested workloads. When the host datapath was the bottleneck, Jool consistently delivered higher throughput and lower RTT than the user-space translators, reflecting reduced context switches and copy overheads in the kernel datapath. When a 1 Gbit/s physical link bounded performance, differences among translators compressed toward the link limit and the gap to Dual-Stack largely disappeared. Platform timing effects were significant: virtualization jitter and clocksource selection (kvm-clock, hpet, tsc) measurably influenced time series smoothness and reported rates, highlighting the need to interpret results relative to same-clock baselines. Overall, the findings support CLAT as a viable choice for IPv6-first deployments where IPv4 addresses are scarce or costly. The study focuses on single-flow TCP and ICMP and leaves CPU cost, tail latency, DNS behavior, and container-orchestrated environments to future work.



\paragraph{Acknowledgments}
I would like to express my sincere gratitude to Dr. Philipp Tiesel for formulating the initial topic and for his mentorship throughout this thesis. His guidance, both technical and methodological, as well as his professional responses to my questions, were invaluable.

I am also deeply grateful to Max Franke for his guidance from the Technische Universität Berlin side. His support with technical issues and university related procedures helped to shape and complete this work.

Finally, I would like to thank the SAP Gardener team for their support during the initial phase of the project, which assisted the early stages of my research.
