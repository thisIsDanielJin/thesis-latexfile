\chapter{Introduction}
The transition from IPv4 to IPv6 has progressed from a long-term goal to a practical necessity. Global IPv4 exhaustion and the growing cost of public IPv4 create pressure to reduce IPv4 dependency while maintaining reachability to IPv4-only services\cite{7737362,LEVIN20141059}. Dual-Stack remains the most widely deployed model because it keeps native behavior for both protocols, but it does not reduce demand for IPv4 addresses and increases operational surface area\cite{rfc4213}. 
Translation-based approaches, particularly NAT64 combined with either DNS64 or client-side translation (CLAT, as in 464XLAT), offer an IPv6-first architecture while still enabling access to IPv4-only endpoints, and have seen broad adoption in mobile networks \cite{rfc6146,rfc6147,rfc6877}. In enterprise and cloud contexts, however, the performance trade-offs between Dual-Stack and 464XLAT remain less well quantified.

\paragraph{Industry collaboration }
This thesis was conducted in close cooperation with SAP SE and was supervised from the industry side by Dr.Philip Tiesel. This collaboration ensured that the research questions addressed are directly aligned with the real-world challenges faced by large-scale platform engineering teams. The primary motivation comes from the architectural decisions required by projects like SAP Gardener, an open-source initiative for managing Kubernetes clusters at scale\cite{gardener_docs}. Within SAP’s Gardeners ecosystem, operators decide between Dual-Stack clusters and IPv6-only clusters. The choice has implications for cost, operational complexity, and performance. 

\paragraph{Problem Statement}

Motivated by this setting, 
the thesis investigates how much latency a CLAT-based path introduces and what performance impacts it has. The study focuses on measuring throughput and round-trip time as fundamental metrics under realistic software implementations and topologies. Specifically, it compares a Dual-Stack baseline 
with three Linux translators that reflect different implementation strategies: Jool (version 4.1.7), a kernel-space translator supporting NAT64 and SIIT \cite{jool_introduction}, Tayga, a stateless user-space NAT64/NAT44 \cite{palrd_tayga_readme,Repas_Farnadi_Lencse_2014} and Tundra, a multi-threaded user-space SIIT/NAT64/CLAT implementation \cite{labuda_tundra_nat64}. Measurements were conducted with 
iperf (version $3.16$) and ping (version iputils $20240117$) across three environments: an AWS deployment, a single-host Ubuntu setup with translators in isolated Linux network namespaces, and a two-host Ubuntu setup connected via Ethernet (specific machine configurations can be found in the Appendix). By limiting the scope to TCP throughput and ICMP RTT, the experiments provide a reproducible view on translation overhead relative to native Dual-Stack, while leaving topics such as tail latency, CPU cost, DNS performance, and application-level behaviors to future work.

The thesis tries to quantify if the measured overhead of CLAT relative to Dual-Stack is acceptable, IPv6-only access with translation becomes an attractive option where IPv4 addresses are scarce or expensive. If not, Dual-Stack remains the safer choice. The following background chapter summarizes the transition mechanisms and Linux implementations that support the experimental design \cite{rfc6877, rfc4241}.


%\section{Thesis Structure}

%This thesis proceeds from motivation and context to reproducible measurement and evidence-based conclusions in a linear fashion. Following the introduction, which frames the problem and outlines the industry motivation in cooperation with SAP, Chapter 2 provides the technical background required to interpret the results. It revisits IPv4 exhaustion and the ongoing transition to IPv6, contrasts Dual-Stack with IPv6-only strategies, and introduces the principles of NAT64/DNS64 together with the role of client-side translation in 464XLAT to explain how IPv4-only endpoints remain reachable from IPv6-only clients. The chapter closes with a description of the software components used in the experiments—Jool, Tayga and Tundra. Chapter 3 presents related work and identifies the specific gap this study addresses by reviewing key studies on IPv6 transition mechanisms, summarizing reported performance characteristics, before pinpointing the lack of publicly available empirical data for the selected open-source implementations in the environments considered here. Chapter 4 then explains the experiment design by introducing the three test environments—an AWS setup, a single Ubuntu host with translators running in separate network namespaces, and a dual-host Ubuntu setup connected via Ethernet with the iperf server on the second machine—documenting how Tayga, Tundra, and Jool are set up, how namespaces and routing are configured, and how measurements are taken with iperf for throughput and ping for RTT, and discussing practical challenges encountered during setup, such as clocksource differences and their effect on timing, together with the steps taken to mitigate them. Chapter 5 presents the test results and evaluation, showing throughput and RTT outcomes per environment, analyzing the translators against the Dual-Stack baseline, synthesizing findings across environments and providing a summary comparison of Jool, Tayga, and Tundra. Finally, Chapter 6 concludes the thesis by summarizing the main findings with respect to the central question, outlining practical implications engineering teams considering IPv6-only with NAT64/DNS64 and CLAT, reflecting on limitations arising from the chosen metrics, software focus, and environments, and suggesting directions for future work, while the document ends with references and an appendix containing configuration files and scripts, and result tables to support reproducibility.
