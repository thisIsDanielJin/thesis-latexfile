\chapter{Introduction}
The transition from IPv4 to IPv6 has shifted from a far concern to an immediate challenge. With IPv4 address exhaustion and rising costs for public IPv4 addresses in cloud environments, organizations face increasing pressure to adopt IPv6. However, complete migration remains complicated by the reality that many applications and services still depend on IPv4 connectivity, whether due to legacy dependencies or third-party integrations.
Two primary approaches have emerged. Dual-Stack maintains both IPv4 and IPv6 connectivity, offering wide compatibility but increasing operational overhead and continuous IPv4 allocation. In contrast, IPv6-only with NAT64/DNS64 and CLAT (464XLAT) enables a smaller IPv4 footprint while maintaining reachability to IPv4 services via translation. While this has worked well in mobile networks, the practical trade-offs for enterprise and cloud workloads remain a subject of debate.
Industry collaboration 
This thesis was conducted in close cooperation with SAP SE and was supervised from the industry side by Dr. Philip Tiesel. This collaboration ensured that the research questions addressed are directly aligned with the real-world challenges faced by large-scale platform engineering teams. The primary motivation comes from the architectural decisions required by projects like SAP Gardener, an open-source initiative for managing Kubernetes clusters at scale. Within SAP’s Gardeners ecosystem, teams decide between Dual-Stack clusters and IPv6-only clusters. The choice has implications for cost, operational complexity, and performance. 

\section{Problem Statement}
While the conceptual trade-offs between Dual-Stack and IPv6-only with 464XLAT are understood, there is a lack of publicly available, empirical data that quantifies the performance overhead of common, open-source translation implementations. Network architects and platform operators are forced to make long-term architectural decisions based on assumptions or data from differing environments, such as mobile networks.
The central problem this thesis addresses is: Does a CLAT implementation create performance disadvantages compared to native Dual-Stack connectivity that would justify the continued reliance on IPv4 infrastructure?
The problem is increased by the fact that performance data from mobile networks, where 464XLAT has seen widespread adoption, may not be directly applicable to data center or cloud environments. Mobile networks operate under different constraints, with specialized hardware and traffic patterns that differ significantly from the high-throughput, low-latency requirements common in enterprise applications.
Moreover, the choice between translation mechanisms is not only about raw performance numbers. Different implementations operate at different layers of the network stack—some in kernel space, others in userspace—each with implications for CPU utilization and integration complexity. Without concrete performance data, teams cannot make informed decisions about which approach best fits their specific use case.
The practical impact of this knowledge gap extends beyond technical considerations. In cloud environments where IPv4 addresses cause direct costs, the performance penalty of translation mechanisms directly influences the economic viability of IPv6-only deployments. If the overhead is minimal, organizations can confidently transition to IPv6-only architectures and reduce their IPv4 footprint. However, if translation introduces significant performance degradation, the operational costs may outweigh the savings from reduced IPv4 usage.


\section{Objectives and Scope}
The objective of this thesis is to produce empirical evidence on the performance trade-offs between native Dual-Stack connectivity and IPv6-only deployments that rely on NAT64/DNS64 with a client-side translator (CLAT) as defined by 464XLAT. In practical terms, the work aims to answer whether a CLAT-based path introduces a measurable performance disadvantage that would argue against replacing Dual-Stack in enterprise and cloud contexts. 
To address this question, the thesis focuses on quantifying the overhead of translation relative to a Dual-Stack baseline using two fundamental network metrics: throughput and round-trip time. Throughput is measured with iperf, and RTT is measured with ping, providing a straightforward and reproducible basis for comparison. The evaluation concentrates on open-source software components that reflect different implementation strategies in the network stack: Jool as a stateful kernel-space NAT64, Tayga as a stateless user-space NAT64, and Tundra as a stateless user-space NAT64. 
The measurements are carried out across three environments: an AWS cloud setup, a single Ubuntu machine where translators run in separate Linux network namespaces on one client host, and a two-host Ubuntu setup connected via Ethernet with the iperf server on the second machine. 
The scope of the work is intentionally narrow to keep the measurements focused and comparable. The metrics are limited to TCP throughput and ICMP RTT. Tail latency, per-packet loss under sustained load, and application-level behaviors are not evaluated. Likewise, the study does not include a systematic analysis of CPU utilization, power consumption, or a cost model. DNS resolution performance itself is not measured. The translators under test are software components typical of general Linux systems. Other transition mechanisms such as SIIT-DC are not considered, since the goal is to compare Dual-Stack with the specific IPv6-only approach built around NAT64/DNS64 and CLAT.
Baselines and roles are defined as follows. The Dual-Stack baseline represents native IPv4/IPv6 connectivity without any translation. The CLAT-based path represents an IPv6-only client that uses a local NAT46 function (CLAT). All translators are placed in separate network namespaces to ensure isolation and to make it possible to observe effects to the corresponding implementation. This layout is kept consistent across the three environments.
Several limitations follow from this design. Because only iperf and ping are used, the results primarily capture bulk data throughput and a basic latency profile. They do not fully characterize behavior under bursty traffic patterns, short flows, or high packet rate scenarios. The choice of software-only translators on Linux narrows the conclusions to deployments that resemble this setup; organizations using hardware offload may see different results. Moreover, the study does not attempt to evaluate security aspects such as application-level gateways or resilience under failures. While these topics are important in practice, including them would shift the focus from the central performance question. A discussion of limitations and their implications is provided in Chapter 6 - Conclusion.
The work is designed to be reproducible. Configurations, scripts, and environment details are documented and collected in the appendix so that others can repeat the measurements or extend them with additional metrics. Chapter 4 - Experiment Design details the setups and methodology, including the namespace configuration on the single-host setup and the role of the iperf server on the second host in the dual Ubuntu environment.


\section{Thesis Structure}

This thesis proceeds from motivation and context to reproducible measurement and evidence-based conclusions in a linear fashion. Following the introduction, which frames the problem and outlines the industry motivation in cooperation with SAP, Chapter 2 provides the technical background required to interpret the results. It revisits IPv4 exhaustion and the ongoing transition to IPv6, contrasts Dual-Stack with IPv6-only strategies, and introduces the principles of NAT64/DNS64 together with the role of client-side translation in 464XLAT to explain how IPv4-only endpoints remain reachable from IPv6-only clients. The chapter closes with a description of the software components used in the experiments—Jool, Tayga and Tundra. Chapter 3 presents related work and identifies the specific gap this study addresses by reviewing key studies on IPv6 transition mechanisms, summarizing reported performance characteristics, before pinpointing the lack of publicly available empirical data for the selected open-source implementations in the environments considered here. Chapter 4 then explains the experiment design by introducing the three test environments—an AWS setup, a single Ubuntu host with translators running in separate network namespaces, and a dual-host Ubuntu setup connected via Ethernet with the iperf server on the second machine—documenting how Tayga, Tundra, and Jool are set up, how namespaces and routing are configured, and how measurements are taken with iperf for throughput and ping for RTT, and discussing practical challenges encountered during setup, such as clocksource differences and their effect on timing, together with the steps taken to mitigate them. Chapter 5 presents the test results and evaluation, showing throughput and RTT outcomes per environment, analyzing the translators against the Dual-Stack baseline, synthesizing findings across environments and providing a summary comparison of Jool, Tayga, and Tundra. Finally, Chapter 6 concludes the thesis by summarizing the main findings with respect to the central question, outlining practical implications engineering teams considering IPv6-only with NAT64/DNS64 and CLAT, reflecting on limitations arising from the chosen metrics, software focus, and environments, and suggesting directions for future work, while the document ends with references and an appendix containing configuration files and scripts, and result tables to support reproducibility.
