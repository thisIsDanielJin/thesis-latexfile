\chapter{Introduction}
The transition from IPv4 to IPv6 has progressed from a long-term goal to a practical necessity. Global IPv4 exhaustion and the growing cost of public IPv4 create pressure to reduce IPv4 dependency while maintaining reachability to IPv4-only services\cite{7737362,LEVIN20141059}. 
Dual-Stack, where both IPv4 and IPv6 run in parallel on the same devices and networks, remains the most widely deployed model because it keeps native behavior for both protocols, but it does not reduce demand for IPv4 addresses and increases operational surface area\cite{rfc4213}. 
Translation-based approaches convert traffic between IPv6 and IPv4. A common variant is NAT64 (translation at the network layer), often paired with DNS64 (synthesizing IPv6 addresses from IPv4 DNS records) or with client-side translation via CLAT, as in 464XLAT (extending NAT64 functionality to end devices). These methods offer an IPv6-first architecture while still enabling access to IPv4-only endpoints and have seen broad adoption in mobile networks\cite{rfc6146,rfc6147,rfc6877}. In enterprise and cloud contexts, however, the performance trade-offs between Dual-Stack and 464XLAT remain less well quantified.


\paragraph{Problem Statement}

Motivated by this setting, 
the thesis investigates how much latency a CLAT-based path introduces and what performance impacts it has. The study focuses on measuring throughput and round-trip time as fundamental metrics under realistic software implementations and topologies. Specifically, it compares a Dual-Stack baseline
with three Linux translators that reflect different implementation strategies: Jool (version 4.1.7), a kernel-space translator supporting NAT64 and SIIT\cite{jool_introduction}, Tayga, a stateless user-space NAT64/NAT44\cite{palrd_tayga_readme,Repas_Farnadi_Lencse_2014} and Tundra, a multi-threaded user-space SIIT/NAT64/CLAT implementation\cite{labuda_tundra_nat64}. Measurements were conducted with
iperf (version $3.16$) and ping (version iputils $20240117$) across three environments: 
an AWS cloud deployment (EC2 m5.large instance, Linux kernel: 6.8.0-1031-aws, 2 vCPUs, RAM: 7.6 Gi and Ubuntu Version: 24.04.2 LTS), a bare-metal host setup (Linux kernel: 5.15.0-87-generic, CPU: 4, RAM: 15 Gi and Ubuntu Version: 22.04.3 LTS) with translators in isolated Linux network namespaces and a bare-metal network setup connected via Ethernet  (specific machine configurations can be found in table \ref{tab:machine_config}). By limiting the scope to TCP throughput and ICMP RTT, the experiments provide a reproducible view on translation overhead relative to native Dual-Stack, while leaving topics such as tail latency, CPU cost, DNS performance, and application-level behaviors to future work.

The thesis tries to quantify if the measured overhead of CLAT relative to Dual-Stack is acceptable, IPv6-only access with translation becomes an attractive option where IPv4 addresses are scarce or expensive. If not, Dual-Stack remains the safer choice. The following background chapter summarizes the transition mechanisms and Linux implementations that support the experimental design\cite{rfc6877, rfc4241}.

\paragraph{Industry collaboration }
This thesis was conducted in close cooperation with SAP SE and was supervised from the industry side by Dr.Philip Tiesel. This collaboration ensured that the research questions addressed are directly aligned with the real-world challenges faced by large-scale platform engineering teams. The primary motivation comes from the architectural decisions required by projects like SAP Gardener, an open-source initiative for managing Kubernetes clusters at scale\cite{gardener_docs}. Within SAP's Gardeners ecosystem, operators decide between Dual-Stack clusters and IPv6-only clusters. The choice has implications for cost, operational complexity, and performance. 



\paragraph{Thesis Structure}
Following the introduction, which frames the problem and outlines the industry motivation, 
Chapter 2 provides the technical background required to interpret the results. It revisits IPv4 exhaustion and the transition to IPv6, 
describes IPv6 transition mechanisms and introduces the principles of client-side translation. 
The chapter closes with a description of the Linux components used in the experiments: Jool, Tayga and Tundra. Chapter 3 presents 
related work and identifies the specific gap this study addresses. 
Chapter 4 then explains the experiment design by introducing the three test environments: an AWS cloud setup, 
a single Ubuntu host with translators running in separate network namespaces, and 
a dual-host Ubuntu setup connected via Ethernet with the iperf server on the second machine. Documenting how Tayga, Tundra, and Jool are set up, 
how namespaces and routing are configured, how measurements are taken for TCP throughput and RTT and discussing practical 
challenges encountered during setup. 
Chapter 5 presents the evaluation, analyzing the translators against the 
Dual-Stack baseline and synthesizing findings across environments. Finally, Chapter 6 
concludes the thesis by summarizing the main findings with respect to the central question and 
suggesting directions for future work, while the document ends with references and an appendix containing configuration files, scripts and 
further plots to support reproducibility.
