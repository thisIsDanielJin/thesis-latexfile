\section*{Zusammenfassung}

Die Erschöpfung des IPv4-Adressraums beschleunigt den Übergang zu IPv6, die dennoch die Erreichbarkeit von ausschließlich IPv4 basierten Diensten sicherstellen müssen. Während Dual-Stack weiterhin das vorherrschende Bereitstellungsmodell ist, bieten übersetzungsbasierte Ansätze, NAT64 mit DNS64 sowie Clientseitige Übersetzung via CLAT im Rahmen von 464XLAT, eine Alternative. Diese Arbeit quantifiziert die Leistungsabstriche zwischen Dual-Stack und CLAT anhand von TCP-Durchsatz und ICMP-Round-Trip-Time (RTT) unter realistischen Linux Implementierungen und Topologien. Drei Übersetzer, Jool (kernel space, SIIT/NAT64), Tayga (zustandslos im user space) und Tundra (multithreaded, user space, SIIT/CLAT), wurden in drei Umgebungen evaluiert: einer AWS-Umgebung, einem einzelnen physischen Server und zwei über Ethernet verbundenen physischen Ubuntu Server. Die Experimente nutzten Linux Namespaces zur Topologie Kontrolle und umfassten IPv6 Basislinien mit ein und zwei Hops, um Hop-Count-Effekte von Übersetzungskosten zu trennen.
Die Ergebnisse zeigen, dass CLAT für die untersuchten Workloads nur einen geringen Leistungsmehraufwand gegenüber Dual-Stack verursacht. War der Host Datenpfad der Engpass, lieferte Jool konsistent höheren Durchsatz und niedrigere RTTs als die user space Übersetzer. Begrenzt hingegen ein physischer 1 Gbit/s Link die Leistung, komprimieren sich die Unterschiede zwischen den Übersetzern in Richtung der Link-Kapazität und die Diskrepanz zu Dual-Stack verschwindet weitgehend. Plattformseitige Timingeinflüsse erwiesen sich als erheblich: Virtualisierungsschwankungen und die Wahl der Clocksource (kvm-clock, hpet, tsc) beeinflussten die Glätte der Zeitreihen und die gemessenen Raten deutlich, was die Notwendigkeit unterstreicht, Ergebnisse stets relativ zur Basislinie mit identischer Clock zu interpretieren. Insgesamt stützen die Befunde CLAT als tragfähige Option für IPv6-first Netze, insbesondere dort, wo IPv4 Adressknappheit oder Kosten gegen Dual-Stack sprechen. Der Fokus lag auf Single-Flow-TCP und ICMP. Aspekte wie CPU-Kosten, Tail-Latenzen, DNS-Verhalten und container-orchestrierte Umgebungen bleiben Gegenstand zukünftiger Arbeiten.
